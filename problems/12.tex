\taskpic{В архиве Снеллиуса найден чертеж оптической схемы. От времени
  чернила выцвели и на чертеже остались видны только три точки ---
  фокус линзы $F$, источник света $S$, точка $L$, принадлежащая
  плоскости тонкой линзы, и часть прямой линии а, соединяющий источник
  света и его изображение $S'$. Из пояснений к чертежу следует, что
  точка $S'$ отстоит от плоскости линзы на расстояние, большим, чем
  точка $S$.  Возможно ли по этим данным восстановить исходную схему?
  Если да, то покажите, как это сделать. Чему равно фокусное
  расстояние линзы?  }{
\begin{tikzpicture}
  \draw[very thick] (0.5,3.5) -- ++(1.5,-1);
  \draw[very thick,dashed] (2,2.5) -- +(1,-2/3);
  \draw[thick] (2.4,1.5) node[left] {$F$}  -- (2.6,1.5);
  \draw[thick] (2.5,1.6)  -- (2.5,1.4);
  \draw[thick] (3.4,1) node[below] {$L$}  -- (3.6,1);
  \draw[thick] (3.5,1.1)  -- (3.5,0.9);
  \draw[fill=black] ($(0.5,3.5)!0.15!(2,2.5)$) circle (0.05)
  node[above] {$S$};
  \draw[fill=black] (2,2.5) circle (0.05) node[above] {$a$};
\end{tikzpicture}
}
% Козел