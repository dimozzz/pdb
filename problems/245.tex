\taskpic{ Тонкое проволочное кольцо массы $M$ стоит на горизонтальной
  плоскости. По кольцу могут скользить без трения две одинковые
  бусинки массой $m$ каждая. В начальный момент времени бусинки
  находятся вблизи верхней точки кольца. Их одновременно отпускают, и
  они начинают двигаться симметрично. При каком отношении масс $n = m/M$
  кольцо оторвётся от плоскости? }
{
  \begin{tikzpicture}
    \draw[thick,interface] (4,0) -- (0,0);
    \draw[thick] (2,1) circle (1cm);
    \draw[fill=black] (1.9,1.99) circle (0.05cm);
    \draw[fill=black] (2.1,1.99) circle (0.05cm);
  \end{tikzpicture}
}
% Россия-2013, 9 класс
