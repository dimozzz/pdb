\taskpic{ Изображённая на рисунке электрическая цепь состоит из двух
  соединённых друг с другом <<чёрных ящиков>>, каждый из которых имеет
  три вывода. При подключении к клеммам $A$ и $C$ омметр показывает
  значение сопротивления $R_{AC}$, при подключении к клеммам $B$ и $D$
  --- значение $R_{BD}$, при подключении к клеммам $A$ и $D$ ---
  значение $R_{AD}$. Что покажет омметр при подключении к клеммам $B$ и
  $C$? Известно, что в <<чёрных ящиках>> находятся только различным
  образом соединённые резисторы.  }  
{
  \begin{tikzpicture}
    \draw[o-o] (0,3) node[blue,right] {$A$} -- (0.6,1.5) -- (0,0)
    node[blue,right] {$B$};
    \draw[o-o] (3,3) node[blue,left] {$C$} -- (2.4,1.5) -- (3,0)
    node[blue,left] {$D$};
    \draw (0.6,1.5) -- (2.4,1.5);
    \draw[fill=black] (1.5,1.5) circle (0.05cm);
    \draw[fill=black] (0.6,1.5) circle (0.3cm);
    \draw[fill=black] (2.4,1.5) circle (0.3cm);
  \end{tikzpicture}
}
