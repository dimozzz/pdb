\task{ Внутри большого мыльного пузыря находится маленький, радиус
  которого в 10 раз меньше. Воздух снаружи откачивают, после чего
  радиус большого пузыря увеличивается в 2 раза. Во сколько раз
  увеличится радиус внутреннего пузыря? Температуру считайте
  постоянной, влиянием силы тяжести можно пренебречь. }
% Квант, Ф1339, 1992-07
