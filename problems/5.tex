\taskpic{В ведре находится смесь воды со льдом массой $m = 10\mbox{
    кг}$. Ведро внесли в комнату и сразу же начали измерять
  температуру смеси. Получившаяся зависимость температуры от времени
  $T(\tau)$ изображена на графике. Удельная теплоемкость воды равна
  $c_{\mbox{\textit{в}}} = 4{,}2 \mbox{
    кДж}/(\mbox{кг}\cdot\mbox{К})$, удельная теплота плавления льда
  $\lambda = 340 \mbox{ кДж/кг}$. Определите массу
  $m_{\mbox{\textit{л}}}$ льда в ведре, когда его внесли в
  комнату. Теплоемкостью ведра
  пренебречь.}{
\begin{tikzpicture}
  \draw[blue,dashed] (0,2) -- (3,2) -- (3,0);
  \draw[thick,->] (0,0) -- ++(3.5,0);
  \draw[thick,->] (0,0) -- ++(0,3.5) node[right=3,fill=white] {\tiny{$T,
    {}^\circ C$}};
  \draw[very thick,red] (0,0) -- (2.5,0) --
  ($(2.5,0)!1.3!(3,2)$);
  \foreach \x in {20,40,60} {
    \draw (\x/20,0.1) -- ++(0,-0.2) node[below=-3] {\tiny{\x}};
  }
  \foreach \y in {1,2,3} {
    \draw (0.1,\y) -- ++(-0.2,0) node[left=-3] {\tiny{\y}};
  }
  \draw (3.5/2,-0.5) node {\tiny{$\tau$, мин}};
\end{tikzpicture}
}
% ММО 1968-1985, 2.36