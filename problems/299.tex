\taskpic{Таракан ползет вдоль стены школьной столовой длиной $30$ м, причем скорость таракана все время меняется. Известен график зависимости величины обратной скорости таракана $v$ от его координаты $x$. Определите по этому графику время, за которое таракан проползет вдоль всей стены.}
{
\begin{tikzpicture}
  \draw[help lines,step=0.5] (0,0) grid (3.15,3.15);  
  \draw[->,thick] (0,0) -- (0,3.3) node[right] {\tiny{$1/v$, с$\cdot$м$^{-1}$}};
  \draw[->,thick] (0,0) -- (3.5,0) node[above] {\tiny{$S$, м}};;
  \draw[very thick,red] (0,3) to[out=-60,in=180] (1.75,1) to
  [out=0,in=-150] (3,1.4);
%  \draw (0,-0.3) node {\tiny{$0$}};
  \draw (0,0.1) -- ++(0,-0.2) node[below] {\tiny{0}};
  \draw (0.5,0.1) -- ++(0,-0.2) node[below] {\tiny{5}};
  \draw (1,0.1) -- ++(0,-0.2) node[below] {\tiny{10}};
  \draw (1.5,0.1) -- ++(0,-0.2) node[below] {\tiny{15}};
  \draw (2,0.1) -- ++(0,-0.2) node[below] {\tiny{20}};
  \draw (2.5,0.1) -- ++(0,-0.2) node[below] {\tiny{25}};
  \draw (3,0.1) -- ++(0,-0.2) node[below] {\tiny{30}};
  \draw(0.1,0) -- ++(-0.2,0) node[left=-3] {\tiny{0}};
  \draw(0.1,0.5) -- ++(-0.2,0) node[left=-3] {\tiny{5}};
  \draw(0.1,1) -- ++(-0.2,0) node[left=-3] {\tiny{10}};
  \draw(0.1,1.5) -- ++(-0.2,0) node[left=-3] {\tiny{15}};
  \draw(0.1,2) -- ++(-0.2,0) node[left=-3] {\tiny{20}};
  \draw(0.1,2.5) -- ++(-0.2,0) node[left=-3] {\tiny{25}};
  \draw(0.1,3) -- ++(-0.2,0) node[left=-3] {\tiny{30}};
\end{tikzpicture}
}

% Богословский, с. 38