\task{ Между городами А и Б по прямой дороге идут навстречу друг другу
два пешехода. Первая встреча между ними произошла точно посередине
дороги, сразу после встречи они с теми же скоростями пошли обратно, а
дойдя до концов дороги снова повернули обратно --- до следующей
встречи и так далее. Вторая встреча произошла через час после первой,
на расстоянии 0,5 км от середины дороги. Через какое время после этого
и где именно произойдёт встреча номер 8? Считайте, что пешеходы
неутомимы. Скорость каждого пешехода меняется только по направлению,
но не по величине. }
% Зильберман, Школьные физические олимпиады, 1.5