\task{ Миниатюрный тигель (печка) для плавки металла имеет
  электронагреватель постоянной мощности $P_0 = 20$ Вт. Нагреватель
  включают и, после того как его температура практически перестает
  увеличиваться, в тигель бросают несколько кусочков олова, общая
  масса которых $m = 80$ г. Олово начинает плавиться. График зависимости
  температуры в тигле от времени представлен на рисунке. Определите
  удельную теплоту плавления олова.}
\vspace{0.5cm}
\begin{center}
  \begin{tikzpicture}
    \begin{axis}[xlabel={$\tau, \text{ мин}$},ylabel={$t, {}^{\circ}C$},xmin=0,xmax=32,ymin=0,ymax=320,minor x tick
      num=5,grid=both,minor y tick num=6,width=10cm,smooth]
      \addplot[thick,blue] coordinates {(0,20) (2,122) (4,190) (6,240)
        (8,270) (10,288) (12,300) (14,304) (15,304)};
      \addplot[thick,blue] coordinates { (15,304) (16,237) (17,230)
        (18,230) (20,230) (22,230) (24,230) (26,230) (28,230) (29,237)
      (30,247) (31,260)};
    \end{axis}
  \end{tikzpicture}
\end{center}