\task{ Запаянная снизу стеклянная пробирка длиной 152 см удерживается в
  вертикальном положении. Верхняя половина пробирки заполнена ртутью,
  нижняя --- воздухом. Воздух медленно нагревают. Какое количество
  теплоты необходимо передать воздуху, чтобы он вытеснил всю ртуть?
  Атмосферное давление $p_0$ составляет 760 мм. рт. ст. Воздух
  считайте двухатомным газом.}