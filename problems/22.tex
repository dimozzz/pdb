 \task{ В чайник с нагревательным элементом мощностью $P$ = 2200 Вт
   налили $V_1 = 1.5 \mbox{ л}$ холодной воды и включили его. когда
   вода закипела, он автоматически отключился. Через $t_1 = 60 \mbox{
     с}$ его снова включили, а еще через $t_2 = 6 \mbox{ с}$ вода
   закипела и чайник выключился. сразу после этого его еще раз
   включили, но сняв крышку. Автоматический выключатель, срабатывающий
   под давлением пара, перестал действовать, и вода из чайника начала
   выкипать. Через $t_3 = 240 \mbox{ с}$ после последнего включения
   измерили объем оставшейся воды. Он оказался равным $V_2 = 1.3
   \mbox{ л}$. Каково значение удельной теплоты парообразования воды
   $L$?  Удельная теплоемкость воды $c = 4200 \mbox{ Дж/(кг*К)}$,
   плотность $\rho = 1000 \mbox{ кг/м}^3$. Теплоемкостью чайника
   пренебречь.}