\task{ Скучающий десятиклассник изготовил вертикальное шероховатое
кольцо радиуса $10 \mbox{ см}$ и нанизал на него бусинку, которую
соединил туго натянутой пружиной с центром кольца. От нечего делать он
стал щёлкать бусинку пальцем в направлении против часовой стрелки
каждый раз, как она проходила нижнюю точку кольца, и через некоторое
время движение стало периодическим. Какова масса бусинки, если с
каждым щелчком бусинке добавляется импульс $0.1 \mbox{ кг·м/с}$, а
единственная точка разворота бусинки находится под углом $60^\circ$ к
вертикали ?  }