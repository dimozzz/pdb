\task{ Два одинаковых лёгких металлических контейнера заполнены
  одинаковым количеством воды и помещены к комнату с постоянной
  температурой воздуха. Шарик, подвешенный на тонкой непроводящей
  верёвке, опускают в один из контейнеров. Масса шарика равна массе
  воды, его плотность намного больше плотности воды. Оба контейнера
  нагреты до температуры кипения воды и начинают остывать. Контейнер с
  шариком остывал до комнатной температуры в $k$ раз больше времени,
  чем контейнер без шарика. Найдите теплоёмкость материала шарика
  $C$. Теплоёмкость воды равна $C_0$. }
% Physics Challengs, TPT, January 2002