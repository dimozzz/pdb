\task{ Кольца могут скользить без трения вдоль закреплённого
  вертикального стержня. В начальный момент все кольца лежат на земле.
  Затем кольцам по очереди, начиная с верхнего, с интервалом времени
  $\tau$ сообщают скорость $v \gg g \tau$ вверх. Встречаясь в полёте,
  кольца слипаются. Кольца бросают вверх до тех пор, пока вся стопка
  колец не упадёт на землю. Когда это произойдёт? Толщиной колец можно
  пренебречь. }
% Город-2005, 10 класс
