\task{ В стоящий на столе калориметр налита вода комнатной
  температуры $t_0$. С большой высоты $h$ в калориметр падают
  одинаковые капли воды той же температуры $t_0$. На уровне
  поверхности воды в калориметре имеется небольшое отверстие, через
  которое вытекает лишняя вода. Какая температура установится в
  калориметре спустя большое время после начала падения капель?
  Удельная теплоемкость воды равна $c$, ускорение свободного падения
  капель равно $g$. Теплоемкостью калориметра, отдачей тепла от его
  стенок и испарением воды можно пренебречь.}
% Мос. олимпиады 2003. 8 класс.
