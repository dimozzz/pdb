\task{ Закреплённая непроводящая тонкостенная однородная сфера
  радиусом $R$ и массой $M$ равномерно заряжена по поверхности зарядом
  $Q$. Из неё вырезают маленький кусочек массой $M/10000$, сжимают его
  в крошечный комочек (не меняя заряд) и помещают в центр
  сферы. Комочек отпускают. Чему будет равна его скорость на большом
  удалении в момент вылета из сферы? }
% Московская городская олимпиада, 2006