\taskpic{Записывая свои воспоминания, барон Мюнхаузен засиделся до
  поздней ночи при свечах. Обе свечи одинаковой длины $l$ он зажег
  одновременно и поставил, как показано на рисунке. Скоро он заметил,
  что тень первой свечи на левой стене неподвижна, а тень второй свечи
  на правой стене укорачивается со скоростью $v$. Через какое время
  барон останется в полной темноте? Считайте, что
  $d_1=d_2=d_3$.}{
\begin{tikzpicture}
  \draw[interface,thick] (3.5,3.5) -- (3.5,0.5) -- (0.5,0.5) --
  (0.5,3.5);
  \draw[thick] (1.4,0.5) rectangle ++(0.2,2);
  \draw[thick] (2.4,0.5) rectangle ++(0.2,2);
  % свечи
  \draw[fill=yellow] (1.5,2.5) to[out=30,in=-80] (1.5,3) to
  [out=-100,in=150] (1.5,2.5);
  \draw[fill=yellow] (2.5,2.5) to[out=30,in=-80] (2.5,3) to
  [out=-100,in=150] (2.5,2.5);
  % разметка
  \draw [blue,dashed] (0.5,2.5) -- ++(3,0);
  \draw[blue,<->] (0.5,1.5) -- (1.4,1.5) node [midway,below] {$d_1$};
  \draw[blue,<->] (1.6,1.5) -- (2.4,1.5) node [midway,below] {$d_2$};
  \draw[blue,<->] (2.6,1.5) -- (3.5,1.5) node [midway,below] {$d_3$};
\end{tikzpicture}
}