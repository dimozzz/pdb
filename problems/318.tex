\taskpic{Колесо радиуса $R$ движется поступательно со скоростью $V_0$.
Первоначально колесо не вращается. Ось колеса может свободно двигаться только вдоль полозьев $AB$,
трение между колесом и поверхностью, а также в оси колеса пренебрежимо мало. Обод колеса равномерно заряжен.
Колесо въезжает в протяженную область, где имеется однородное магнитное поле индукции $B$,
параллельное оси колеса (см. рисунок).\\
Каков должен быть заряд колеса, чтобы на большом расстоянии от границы раздела $OO'$ колесо покатилось без проскальзывания?
Масса колеса равна $M$ и сосредоточена в ободе. Излучением пренебречь.}
%#TODO pic
{}
%Санкт-Петербургская городская олимпиада по физике, 2007 год