\taskpic[3cm]{ В закрытом сосуде с жёсткими стенками ёмкостью $V=1$ л
  находятся $V_1=0{,}8$ л воды и сухой воздух при атмосферном давлении
  $p_{0}$ и температуре $T_1=30^{\circ}$C. Сосуд представляет собой
  перевёрнутый основанием вверх конус. Поверх воды налит тонкий слой
  машинного масла, отделяющий воду от воздуха. Сосуд охлаждают до
  температуры $T_2=-30^{\circ}$C, при этом вся вода
  замерзает. Плотность воды $\rho_1=1$ г/см$^3$, плотность льда
  $\rho_2=0{,}9$ г/см$^3$. Определите давление воздуха надо льдом. }
{
  \begin{tikzpicture}
    \draw[very thick] (0,3) -- (1.5,0) -- (3,3) -- cycle;
    \draw[thick] (0.35,2.3) -- (2.65,2.3);
    \draw (1.5,2.6) node {\textit{воздух}};
    \draw (1.5,1.9) node {\textit{вода}};
  \end{tikzpicture}
}