\task{ В калориметр налиты $m_1=0{,}5$ кг воды при температуре
  $t_1=15^{\circ}$C. В воду опускают кусок льда с массой $m_2=0{,}5$
  кг, имеющий температуру $t_2=-10^{\circ}$C. Найти температуру смеси
  после установления теплового равновесия. Удельная теплоёмкость воды
  $c_1=4200$ Дж/(кг$\cdot {}^{\circ}$C), удельная теплоёмкость льда
  $c_2=2100$ Дж/(кг$\cdot {}^{\circ}$C), удельная теплота плавления
  льда $\lambda=3{,}3\cdot 10^5$ Дж/кг. }
% Слободецкий-Орлов, №58 (Союз-1968)