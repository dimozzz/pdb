\taskpic{Изогнутая проволока состоит из большого количества полуокружностей радиуса $R$. 
По проволоке может без трения скользить маленькая бусинка. Проволока наклонена под углом $\alpha = \frac{1}{120}$ радиана к горизонту. 
Первоначально бусинка находится на верхнем участке проволоки (см. рисунок). \\
Бусинку отпускают. Известно, что к тому моменту, когда бусинка достигла нижнего конца проволоки,
она 150 раз испытала состояние невесомости.\\
Найти длину проволоки. Ускорение свободного падения $g$.}
%#TODO pic
{}
%Санкт-Петербургская городская олимпиада, 2002 год