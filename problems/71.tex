\task{ К телу массой $m = 20$~г, лежащему на гладком горизонтальном
  полу, привязаны две одинаковые упругие нити жёсткостью $k =
  10^4$~Н/м. Одна нить прикреплена к стене, свободный конец второй
  нити начинают тянуть в горизонтальном направлении со скоростью $v =
  20$~м/с. Какая нить порвётся, если разрыв каждой нити происходит при
  абсолютном удлинении $\Delta l_{пр} = 5$~мм? Считать, что закон
  Гука выполняется для нитей вплоть до их разрыва; трения нет.  }
