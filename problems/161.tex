\task{ Поверхность образца породы начинает разрушаться, если на нее
  воздействует давление большее некоторой определенной
  величины. Модель лунохода в земных испытаниях имела наибольшую
  массу, при которой она двигаясь по образцу данной породы, еще не
  разрушала ее. Во сколько раз размеры оригинала лунохода могут
  отличаться от земной его модели, чтобы луноход не разрушал эту
  породу на Луне? Луноход и его модель изготовлены из одних и тех же
  материалов.  На поверхности Земли $g_{\mbox{З}} =9,8$ Н/кг, на
  поверхности Луны $g_{\mbox{Л}}= 1,6$ Н/кг.}
% Санкт-Петербургская городская олимпиада, городской тур, 2002 год