\task{
  Предположим, что создан материал с необычной зависимостью
  коэффициента теплопроводности $k$ от температуры. Пластину из такого
  материала поместили между двумя стенками вплотную к ним. Температуры
  стенок поддерживаются постоянными: $T_1 = 160 K$ и $T_2 = 500 K$
  соответственно. Какой тепловой поток установится между стенками,
  если толщина пластины $d=1 \mbox{ см}$, а её площадь $S= 100 \mbox{
    см}^2$? Найдите температуру в среднем продольном сечении пластины
  ($x=d/2$). \\
\textit{Указание.} Тепловой поток $P$ сквозь тонкий слой вещества,
площадь которого равна $S$, а толщина $dx$, равный количеству теплоты,
проходящему через этот слой в единицу времени, прямо пропорционален
разности значений температуры его поверхностей $dT$ и обратно
пропорционален его толщине: $P = - k S \dfrac{dT}{dx}$, где $k$ ---
коэффициент теплопроводности вещества. 
}

\begin{center}
  \begin{tikzpicture}
    \begin{axis}[ymin=0,ymax=2,xmin=0,xmax=600,
      x=0.01cm,
      y=2cm,
      minor tick num=3,
      xlabel={$T$, K},
      ylabel={$k$, Вт/м $\cdot$ K},
      grid=both
      ]
    \coordinate (a) at (80,180);
    \coordinate (b) at (530,120);
    \coordinate (c) at (180,250);
    \coordinate (d) at (320,0);
    \draw[very thick,red] (a) .. controls (c) and (d) .. (b);
  \end{axis}
\end{tikzpicture}
\end{center}
% Козел
