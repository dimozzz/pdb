\taskpic{В схеме, изображённой на рисунке, периодически (с периодом
  $3\tau$) повторяют следующий процесс: ключ замыкают на время $\tau$
  и размыкают на время $2\tau$, причём $\tau \ll RC$ и напряжение на
  конденсаторе за это время изменяется незначительно. Через достаточно
  большое число повторений напряжение на конденсаторе становится
  практически постоянным, совершая лишь незначительные колебания около
  своего среднего значения. Найдите среднюю тепловую мощность,
  выделяющуюся в резисторе $2R$ в установившемся режиме. Все элементы
  можно считать идеальными, их параметры указаны на рисунке.}
{
  \begin{tikzpicture}[circuit ee IEC,scale=0.8]
    \node[contact] (A) at (2,2) {};
    \node[contact] (B) at (2,0) {};
    \draw[thick] (A) to[resistor={info'=$R$}] (0,2)
    to[battery={info=$\mathcal{E}$} ] (0,0) -- (4,0)
    to[capacitor={info=$C$}] (4,2) to[resistor={info'=$2R$}] (A);
    \draw[thick] (A) to[make contact] (B);
  \end{tikzpicture}
}
% МФТИ, сборник задач для абитуриентов; летняя школа "Рысь", 2012
