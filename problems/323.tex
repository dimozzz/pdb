\task{ Правый конец металлического стержня длиной 1~м погружен в
  кипящий ацетон. На расстоянии 47~см от левого конца стержня лежит
  маленький кристалл нафталина. Левый конец стержня погрузили в
  кипящую воду. Какая доля ацетона выкипит, пока расплавится весь
  нафталин? Считайте, что вся теплопередача происходит только через
  стержень, а поток тепловой энергии через тонкий слой прямо
  пропорционален разности температур на торцах слоя. Количество
  кипящей воды в сосуде очень велико, кипение поддерживается.
  Температура кипения ацетона $56{,}2^\circ\mbox{С}$, температура
  плавления нафталина $80{,}3^\circ\mbox{С}$. }
