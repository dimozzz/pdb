\task{ Стержень вытаскивают из трубы, имеющей диаметр, несколько
  больший диаметра стержня. В зазор между стержнем и трубой попадает
  песчинка, имеющая форму параллелепипеда (отношение $a/b=0.1$). При
  каком значении коэффициента трения между песчинкой и поверхностями
  стержня и трубы стержень не удастся вытащить из трубы? Считать, что
  коэффициент трения между трубой и стержнем пренебрежимо мал.  }
\begin{center}
  \parbox{6cm}{
  \begin{tikzpicture}
    \draw[thick,interface] (4,0) -- (0,0);
    \draw[thick,interface] (0,2) -- (4,2);
    \draw[very thick,fill=gray!20] (0.5,0.1) rectangle (4.5,1.9)
    node[midway] {стержень};
    \draw[thick,->] (4.6,1) -- (5.5,1);
  \end{tikzpicture}}
  \hspace{2cm}
  \parbox{6cm}{
  \begin{tikzpicture}
    \draw[thick,interface] (0,2) -- (4,2);
    \draw[very thick,fill=gray!20] (0,0) rectangle (4,1) node[midway]
    {стержень}; 
    \begin{scope}[rotate around={36:(1,1)}]
      \draw[thick] (1,1) rectangle (2,1.5);
      \draw[blue] (1.5,0.85) node {$a$};
      \draw[blue] (2.15,1.25) node {$b$};
    \end{scope}
  \end{tikzpicture}}
\end{center}
% Квант, 1980, №11 (Асламазов)