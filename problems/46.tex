\task{В электрических цепях часто используются двухпозиционные
  переключатели, в которых средний контакт 0 замыкается либо на
  контакт 1, либо на контакт 2. Из двух таких переключателей, двух
  одинаковых лампочек и батарейки составьте схему, которая работает
  следующим образом. При четырёх различных положения пары ключей
  реализуются четыре режима. 

  \begin{enumerate}
        \item Обе лампочки не горят;
        \item Одна лампочка не горит, а другая горит в полный накал;
        \item Обе лампочки горят в полный накал;
        \item Обе лампочки горят в полнакала.
  \end{enumerate}

Кроме того, ни при каких положениях переключателей в схеме не должно
быть короткого замыкания батарейки. Э.д.с. батарейки равняется
номинальному напряжению лампочки. Под горением в полнакала понимается
режим, при котором \textit{ток} через лампочку равен половине от
номинального.}
