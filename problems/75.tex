\taskpic{ Конденсаторы $C$, $2C$ и $3C$ соединены между собой, как
  показано на рисунке. Между свободными выводами конденсаторов $C$ и
  $2C$ подключен резистор $3R$, между свободными выводами
  конденсаторов $C$ и $3C$ --- резистор $2R$, между оставшимися ---
  резистор $R$. В начальный момент конденсатор $2C$ заряжен до
  напряжения $U$, остальные конденсаторы не заряжены. Какое количество
  тепла выделится за большое время на резисторе $R$?  }{
\begin{circuitikz}[scale=0.9]
  \draw (0,0) to [generic, l=$3R$] (0,2) -- (0,3.5) to [generic, l=$2R$]
  (2.6,3.5) -- (2.6,2) to [generic, l=$R$] (2.6,0) -- (0,0) (0,2) to [C,
  l=$C$] (1.15,2) (1.15,2) to [C, l=$3C$] (2.6,2);
  \draw (1.15,2) to [C,l=$2C$] (1.15,0); 
\end{circuitikz}
}
% Сорос, 2000