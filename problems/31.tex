\task{В проводящую жидкость с диэлектрической проницаемостью
  $\varepsilon$ на расстоянии $L$ друг от друга помещают два маленьких
  металлических шарика, заряженных до зарядов $Q$ и $-Q$. Сила
  сопротивления, действующая на шарики в жидкости, пропорциональна
  квадрату скорости $F_c = \alpha v^2$, причем $\alpha \gg m/r$, где
  $r$~---~радиус шариков, $m$~---~их масса. На какое расстояние могут
  сблизиться шарики? Удельное сопротивление жидкости~---~$\rho$. }