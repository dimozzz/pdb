\task{ В безветренную погоду на озере вблизи берега покоится лодка. На
ближнем к берегу её конце сидит мальчик. Определите, сможет ли мальчик
заставить лодку подойти вплотную к берегу, если известно, что масса
мальчика $50 \mbox{ кг}$, его максимальная скорость относительно лодки
$2 \mbox{ м/с}$ , а расстояние от лодки до берега $1.5 \mbox{
м}$. Считайте, что сила сопротивления воды движению лодки
прямопропорциональна её скорости с известным коэффициентом $100 \mbox{
кг/с}$ .  }

% Ответ: Не сможет.