% Квант, 1981-12
\taskpic{Точечный предмет $S$ находится на оси полого конуса с
  зеркальной внутренней поверхностью. С помощью линзы Л на экране Э
  получают изображение предмета, создаваемое лучами, однократно
  отраженными от зеркальной поверхности конуса (прямые лучи от
  предмета на линзу не попадают). Что произойдет с изображением, если
  линзу закрыть диафрагмой, такой как на рис. $A$ и такой, как на
  рис. $B$?}{
  \begin{tikzpicture}
    \draw[very thick,dashed] (0,2) -- (4,2);
    \draw[thick,interface] (2.4,3) -- (3.8,2) -- (2.4,1);
    \draw[thick] (2.7,2.45) arc (120:240:0.5cm);
    \draw[thick] (0.6,0.5) -- (0.6,3.5) node [above] {Э};
    \draw[thick, <->] (1.3,0.5) -- (1.3,3.5) node [above] {Л};
    \draw[line width=0.1cm] (0.9,-0.5) circle (0.5cm) node [below=0.5cm]
    {$A$};
    \def\inner {(2.875,-0.01) arc (105:75:0.5cm) -- ++(0,-0.98) arc (-75:-105:0.5cm) --
      ++(0,0.98)};
    \fill[even odd rule] (3,-0.5) circle (0.5cm) node[below=0.55cm] {$B$}
    (2.9,-1.1) \inner;
    \draw[blue] (3,1.75) -- (3,2.25);
    \draw[blue] (2.8,1.8) -- (3.2,2.2);
    \draw[blue] (2.8,2.2) -- (3.2,1.8);
    \filldraw[blue] (3,2) circle (0.1cm) node[black,below=0.15cm] {\tiny{$S$}};
  \end{tikzpicture}
}
