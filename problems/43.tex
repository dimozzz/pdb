\taskpic{С помощью системы концентрических зеркал на экране получено
  изображение Солнца. Радиусы зеркал $R_1 = 12$ см, $R_2 = 30$
  см. Каково должно быть фокусное расстояние тонкой линзы, чтобы с её
  помощью получалось изображение Солнца такого же размера?}{
\begin{tikzpicture}
  \draw[thick,interface] ({3+cos(110)},{2+sin(110)}) arc (110:250:1);
  \draw[thick,interface] ({3+1.5*cos(160)},{2+1.5*sin(160)}) arc
  (160:110:1.5);
  \draw[thick,interface] ({3+1.5*cos(250)},{2+1.5*sin(250)}) arc
  (250:200:1.5);
  \draw[thick,->] (0,2) -- (1.2,2);
  \draw[thick,->] (0,2.4) -- (1.2,2.4);
  \draw[thick,->] (0,1.6) -- (1.2,1.6);
  \draw[blue,->] (3,2) -- ({3+cos(140)},{2+sin(140)}) node[near
  start,above] {$R_1$};
  \draw[blue,->] (3,2) -- ({3+1.5*cos(220)},{2+1.5*sin(220)})
  node[near start,below] {$R_2$};
\end{tikzpicture}
}
% Квант, 1981-09, Ф677