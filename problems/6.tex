\taskpic{В схеме, изображенной на рисунке, сопротивления всех
  резисторов одинаковы и равны $R$. Напряжение на обозначенных клеммах
  равно $U$. Определите силу тока $I$ в подводящих проводах, если их
  сопротивлением можно пренебречь.}
{
  \begin{tikzpicture}[circuit ee IEC,scale=0.9]
    \node[contact,info={$A$}] (A) at (0,0) {};
    \node[contact,info={$B$}] (B) at (4,0) {};
    \node[contact,info={$C$}] (C) at (2,2) {};
    \node[contact,info'={$D$}] (D) at (2,-2) {};
    \node[contact] (O) at (2,0) {};
    \draw (A) to[resistor] (O) to [resistor] (C) to[resistor] (B)
    to[resistor] (D) to[resistor] (A);
    \draw (A) to[resistor] (C) (O) to[resistor] (D) (O) to[resistor]
    (B);
    \draw[-o] (A) -- (0,-1);
    \draw[-o] (O) to[out=235,in=0] (1,-2);
    \draw[<->] (0,-1) -- (1,-2) node[midway,fill=white] {$U$};
  \end{tikzpicture}
}
% ММО 1968-1985, 3.25
