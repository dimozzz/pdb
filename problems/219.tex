\task{ Рыбак пересекает реку ширины $d$ дважды. В первый раз он
  пересекает её, затрачивая наименьшее время $T$. Во второй раз он
  пересекает реку таким образом, чтобы его расстояние, на которое
  сносит лодку относительно противоположного берега, оказалось
  минимальным. При этом время, которое он тратит на пересечение,
  оказывается равным $3T$. Найдите скорость течения реки. }
% Physics Challenges, TPT, September 2006
% внимание: в этой задаче два решения, поскольку лодку может сносить в
% обе стороны