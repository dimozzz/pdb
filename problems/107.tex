\taskpic{ По реке со скоростью $v$ плывут мелкие льдины, которые
  равномерно распределяются по поверхности воды, покрывая ее $n$-ю
  часть. В некотором месте реки образовался затор. В заторе льдины
  полностью покрывают поверхность воды, не нагромождаясь друг на
  друга.  Какая сила действует на 1 м ледяной границы между водой и
  сплошным льдом в заторе со стороны останавливающихся льдин?
  Плотность льда $\rho= 0{,}91 \cdot 10^3 \text{ кг/м}^3$; толщина $h =
  20$ см; скорость реки $v = 0{,}72$ км/ч; плывущие льдины покрывают $n =
  0{,}1$ часть поверхности воды. }{
  \begin{tikzpicture}
    \draw[pattern=north east lines] (0,3) rectangle (4,3.5);
    \draw[pattern=crosshatch] (0,0.5) rectangle (3.5,3);
    \draw[pattern=north east lines] (0,0) rectangle (4,0.5);
    \draw[pattern=north west lines]  (3.5,0.5) rectangle (4,3);
    \draw[very thick,->] (1.5,1.75) -- (2.5,1.75) node[fill=white,midway,above=0.15cm] {$\vec{v}$};
  \end{tikzpicture}
}
