\task{ На некоторой планете может быть реализован следующий
  эксперимент. При плоских колебаниях математического маятника длиной
  $L = 3 \mbox{ м}$ максимальная сила натяжения нити отличается от
  минимальной в $k = 4$ раза, если максимальный угол отклонения равен
  значению угла $\alpha$. Такой же угол $\alpha$ с вертикалью образует
  нить маятника, если она вращается с периодом $T = 4 \mbox{ с}$ вокруг
  вертикальной оси, проходящей через точку подвеса. Определите
  ускорение свободного падения на данной палате.  \\
  \textit{Примечание.} Частота колебаний математического маятника
    зависит только от длины подвеса и ускорения свободного падения:
  $\omega = \sqrt{ g/L }$.  }