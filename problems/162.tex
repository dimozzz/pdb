\task{ Горячий суп, налитый доверху в большую тарелку, охлаждается до
  температуры, при которой его можно есть без риска обжечься, за время
  $t = 20$ мин. Через какое время можно будет есть суп с той же
  начальной температурой, если разлить его по маленьким тарелкам,
  которые также заполнены доверху и подобны большой? Известно, что суп
  из большой тарелки помещается в n = 8 маленьких, и что количество
  тепла, отдаваемое в единицу времени с единицы поверхности каждой
  тарелки, пропорционально разности температур супа и окружающей
  среды.  }
% Московские физические олимпиады, 2000, 2.27