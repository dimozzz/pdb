\task{ На расстоянии $d$ от тонкой собирающей линзы вдоль её главной
оптической оси расположена тонкая короткая палочка. Длина её
действительного изображения, даваемого линзой, в $k$ раз больше длины
палочки. Во сколько изменится длина изображения, если сдвинуть палочку
параллельно оси на $\delta d$ дальше от линзы? }