
\task { На тонкий тяжелый обруч массой 2 кг, имеющий неподвижную ось
  вращения, намотана длинная нить, под обручем находится утрамбованный
  песок. К нити прикреплен груз массой 3 кг. Когда нить была полностью
  намотана на обруч, высота груза над поверхностью песка составляла 2
  м. Обруч отпускают, при этом груз падает, а веревка
  разматывается. Определить, на какую максимальную высоту поднимется
  груз после того, как, упав в песок, и, некоторое время простояв на
  нем, вновь увлечется нитью вверх.\\ \textit{Указание:} считать, что
  кинетическая энергия тела и вращающегося вокруг оси обруча
  определяется по формуле $mv^2/2$ , где в случае обруча $v$ ---
  скорость точек обода. }
% Киров
