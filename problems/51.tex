\task{Для исследования свойств нелинейного резистора был произведен
  ряд экспериментов. Вначале была исследована зависимость
  сопротивления резистора от температуры. При повышении температуры до
  значения $t_1=100^{\circ}$C мгновенно происходил скачок сопротивления от
  величины $R_1=50$~Ом до величины $R_2=100$~Ом, при охлаждении
  обратный скачок происходил при температуре $t_2=99^{\circ}$C. Во
  втором опыте к резистору приложили постоянное напряжение $U_1=60$~В,
  при котором его температура оказалась равной
  $t_3=80^{\circ}$C. Наконец, когда к резистору приложили постоянное
  напряжение $U_2=80$~В, в цепи возникли самопроизвольные колебания
  тока. Определите период этих колебаний а также максимальное значение
  тока. Температура воздуха в лаборатории постоянна и равна
  $t_0=20^{\circ}$C. Теплоотдача от резистора пропорциональна разности
  температур резистора и окружающего воздуха. Теплоемкость резистора
  $C = 3$~Дж/К.}
