\task{ Большое число одинаковых плоских монет уложили плоскими
  сторонами вплотную друг к другу, разделив их круглыми кусочками
  бумаги, совпадающими по диаметру с монетами. Получившийся длинный
  цилиндр завернули бумагой в два слоя. Один из торцов этого цилиндра
  касается термостата, имеющего постоянную температуру $Т_1$. Ближайшую к
  термостату монету и сам термостат отделяет кусочек бумаги толщиной
  $h$. Сам цилиндр находится в воздухе, температура которого
  $Т_0$. Теплопроводность монет намного больше, чем теплопроводность
  бумаги. Диаметр монеты $d$, толщина монеты $Н$. Толщина слоя бумаги $h$
  ($h \ll d$). Теплопроводность материала бумаги --- $k$. Со временем
  установилось стационарное распределение температуры. Какое
  количество тепла получает цилиндр из монет от термостата в единицу
  времени?
\\
\textit{  Указание.} Тепловой поток $P$ сквозь тонкий слой вещества, площадь
  которого $S$, а толщина $dx$, равный количеству теплоты, проходящему
  сквозь этот слой в единицу времени, прямо пропорционален разности
  значений температуры его поверхностей $dT$ и обратно пропорционален
  его толщине:
  $P = -k S(dT/dx)$, где $k$ - коэффициент теплопроводности вещества.
}
