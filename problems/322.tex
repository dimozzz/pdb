\task{ Два одинаковых провощящих шарика радиуса $R$ соединены длинной
  натянутой тонкой проволочкой длины $L$ ($L\gg R$). Систему внесли в
  однородное электрическое поле $E_0$, направленное вдоль
  проволочки. Какой заряд перетечёт по проволочке? Какое количество
  тепла выделится в сопротивлении проволочки? }

% Зильберман, стр. 100
% Довольно сложная задача. Но можно догадаться до простого решения ---
% заряд будет перетекать, пока разность потенциалов между шариками не
% станет равна LE. Количество тепла посчитать просто, т.к. разность
% потенциалов линейно зависит от протекшего заряда. 
