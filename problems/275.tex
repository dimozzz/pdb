\task{ На улице идет сильный дождь. Его капли массой $m = 0.1$~г
  падают вертикально со скоростью $v_1 = 3$~м/с, причем в каждом
  кубометре воздуха содержится $N = 100$ капель. Школьник хочет
  перебежать из своего дома к приятелю в соседний дом, который
  находится на расстоянии $L = 50$~м, и при этом вымокнуть как можно
  меньше. Скорость бега может быть любой, но не выше $v_2 =
  10$~м/с. Какова минимальная масса воды, которая попадет на школьника
  во время забега, если площадь проекции его тела на горизонтальную
  плоскость равна $S_1 = 0.16$~м$^2$, а на вертикальную --- $S_2 =
  0.45$~м$^2$.}
% Мос. олимпиады 2001. 8 класс.
