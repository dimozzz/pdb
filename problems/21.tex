 \taskpic{ В чёрном ящике находится резистор, имеющий постоянное
   сопротивление и нелинейный элемент, которые могут быть включены как
   последовательно, так и параллельно. Найдите сопротивление
   резистора. Какой нелинейный элемент может находиться внутри чёрного
   ящика? ВАХ для последовательно и параллельно включенных элементов
   представлены на рисунке.}
{
 \begin{tikzpicture}[/pgfplots/axis labels at tip/.style={
    xlabel style={at={(current axis.right of origin)}, 
      yshift=2 ex, anchor=east,fill=white}, 
    ylabel style={at={(current axis.above origin)}, yshift=1.5ex,
      anchor=center}}] 
    \begin{axis}[
      width=4.7cm,
      xmin=0,xmax=3,ymin=0,ymax=0.24,
      axis x line=bottom,
      axis y line=middle,
      minor tick num=3,
      axis labels at tip,
      xlabel={$U$, В},
      ylabel={$I$, А},
      tick label style={font=\tiny},
      label style={font=\tiny},
      grid=both
      ]
      \draw[very thick,red] (0,0) to[out=45,in=185] (290,130);
      \draw[very thick,red] (0,0) to[out=70,in=230] (100,210);
    \end{axis}
  \end{tikzpicture}
}
