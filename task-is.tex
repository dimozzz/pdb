%необходимые пакеты
\usepackage{tabularx}
\usepackage{makecell}

% счётчик задач
\newcounter{notask}
\setcounter{notask}{1}

% \task{УСЛОВИЕ ЗАДАЧИ}
% задача без картинки
% оформлена как таблица с двумя колонками
% ширина первой колонки (номер столбца) фиксирована, 0.3cm
% ширина второй колонки автоматически рассчитывается из ширины
% страницы (с учётом всевозможных отступов)

\newcommand{\task}[1]{
  \begin{tabularx}{\textwidth}{|c|X|}
    \cline{1-2}
    \makecell*[{{p{0.3cm}}}]{ \centering \arabic{notask}} &
    \makecell*[{{p{\hsize}}}]{ #1 } \\
    \cline{1-2}
  \end{tabularx}

  \vspace{-1pt}

  \addtocounter{notask}{1}
}


% \taskpic[ШИРИНА КАРТИНКИ]{УСЛОВИЕ ЗАДАЧИ}{КАРТИНКА}
% задача с картинкой
% оформлена как таблица с тремя колонками
% первый аргумент - необязательный, по умолчанию ширина картинки равна
% 4cm, но можно выставить свою
% ширина второй колонки (условие задачи) рассчитывается из ширины
% страницы и ширины картинки

\newcommand{\taskpic}[3][4cm]{
  \begin{tabularx}{\textwidth}{|c|X|c|}
    \cline{1-3}
    \makecell*[{{p{0.3cm}}}]{ \centering \arabic{notask}} &
    \makecell*[{{p{\hsize}}}]{ #2 } &
    \makecell*[{{p{#1}}}]{ \centering #3} \\
    \cline{1-3}
  \end{tabularx}

  \vspace{-1pt}

  \addtocounter{notask}{1}
}

